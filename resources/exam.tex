% Author: Carsten Gips <carsten.gips@fh-bielefeld.de>
% Copyright: (c) 2016-2018 Carsten Gips
% License: MIT


%------------------------------------- Packages ----------------
\usepackage{xcolor}
\usepackage[absolute]{textpos}
\usepackage{colortbl}
\usepackage{wasysym}    % \Square
\usepackage{ifoddpage}
%------------------------------------- Packages ----------------

%------------------------------------- Settings ---------------------
\extrawidth{.5in}
%------------------------------------- Settings ---------------------

%------------------------------------- CMDs ---------------------
\newcommand{\hsfont}    {\fontfamily{phv}\fontseries{m}\fontshape{n}\selectfont}
\newcommand{\hsheadfont}{\fontfamily{phv}\fontseries{b}\fontshape{n}\selectfont}

%% TODO to be replaced by some fenced Div or inline Span and lua filter %%%%%%%%%%%%%%%
%% may be `:::columns```, which a filter could convert to minipages?
\newcommand{\minipagebegin}{\begin{minipage}}
\newcommand{\minipageend}{\end{minipage}}
%% use `:::center``` plus filter from lectures
\newcommand{\centerbegin}{\begin{center}}
\newcommand{\centerend}{\end{center}}

%% instead of `\hinweis ...` we could use `[...]{.hinweis}` plus some lua filter ...
\newcommand{\hinweis}{\emph{Hinweis}:\xspace}
%% instead of `\x{...}` we could use `[...]{.x}` plus some lua filter ...
\newcommand{\x}[1]{\ifprintanswers{\color{red}\bfseries#1\xspace}\fi}
%%%%%%%%%%%%%%%%%%%%%%%%%%%%%%%%%%%%%%%%%%%%%%%%%%%%%%%%%%%%%%%%%%%%%%%%%%%%%%%%%%%%%%%

\newcommand{\Fortsetzung}{\begin{textblock*}{54mm}(80mm,276mm)\textsl{\textbf{Fortsetzung nächste Seite}}\end{textblock*}}
\newcommand{\Leerseite}{\newpage\centerline{\textsl{\textbf{Leerseite}}}\newpage}
%------------------------------------- CMDs ---------------------

%------------------------------------- Listings ---------------------
%% settings for pandoc option `--listings`
\usepackage{listings}  
\lstset{basicstyle=\footnotesize\ttfamily\mdseries, xleftmargin=\bigskipamount, keywordstyle=\bfseries\color{dkblue}, identifierstyle=\ttfamily, commentstyle=\bfseries\color{gray}\textsl, stringstyle=\color{magenta}\upshape, emphstyle=\color{red}, emphstyle={[2]\color{blue}}, texcl=false, showspaces=false, showstringspaces=false, numbers=left, numberstyle=\footnotesize, breaklines=true, tabsize=4, backgroundcolor=\color{listinggray}, frame=shadowbox}

\newcommand{\code}[1]{\textcolor{dkgreen}{\texttt{\textbf{#1}}}\xspace}
%------------------------------------- Listings ---------------------

%------------------------------------- Tables (left column gray background) --------------------------------
%% TODO remove obsolete definitions, use fenced Divs plus lua filter instead of TeX environments(?) %%%%%%%%%%%%%%%
\newenvironment{streifenenv}
{
    \smallskip
    \begin{tabular}{>{\columncolor{headcolor}}p{1pt}p{0.94\textwidth}}
        &
        \begin{minipage}{0.94\textwidth}
}
{
        \end{minipage}
    \end{tabular}
    \smallskip
}
\newcommand{\streifenbegin}{\begin{streifenenv}}
\newcommand{\streifenend}{\end{streifenenv}}


\newenvironment{streifenSolution}[1][10mm]%
{
    \streifenbegin
    \begin{solution}[#1]
}
{
    \end{solution}
    \streifenend
}
\newcommand{\solutionbegin}{\begin{streifenSolution}}
\newcommand{\solutionend}{\end{streifenSolution}}


\newenvironment{streifenMC}[2]
{
    \streifenbegin
    \renewcommand{\arraystretch}{1.2}
    \begin{tabular}{ccp{0.82\textwidth}}
    \textbf{#1} & \textbf{#2} \\
}
{
    \end{tabular}
    \renewcommand{\arraystretch}{1.0}
    \streifenend
}
\newcommand{\antwort}{\ifprintanswers \ensuremath{\blacksquare} \else \ensuremath{\Box} \fi}
\newcommand{\wahr}{\antwort & \ensuremath{\Box} & }
\newcommand{\falsch}{\ensuremath{\Box} & \antwort & }


\newenvironment{streifenTabular}[1]
{
    \streifenbegin
    \ifprintanswers \renewcommand{\arraystretch}{0.6} \else \renewcommand{\arraystretch}{2.0} \fi
    \begin{tabular}{#1}
}
{
    \end{tabular}
    \renewcommand{\arraystretch}{1.0}
    \streifenend
}
%------------------------------------- Tables (left column gray background) --------------------------------

%------------------------------------- Answers --------------------------------
\CorrectChoiceEmphasis{\color{red}\bfseries}
\checkboxchar{\ensuremath{\Box}}
\checkedchar{\ensuremath{\blacksquare}}
\shadedsolutions
\definecolor{SolutionColor}{rgb}{0.9,0.8,0.8}
%------------------------------------- Answers --------------------------------

%------------------------------------- Grading Table --------------------------------
\hqword{\textbf{Aufgabe}}
\hpword{\textbf{Punkte}}
\hsword{\raisebox{-1mm}[6mm][4mm]{\textbf{Erreicht}}}
\htword{\raisebox{-1mm}[6mm][4mm]{\textcolor{headcolor}{\LARGE\ensuremath{\;\;\;\pmb{\Sigma}\;\;\;}}}}
%------------------------------------- Grading Table --------------------------------

%------------------------------------- Custom Title Page ----------------
\renewcommand{\maketitle}{} % we use "coverpage" from the exam package instead
\renewcommand{\tableofcontents}{} % we have to use --toc to compile the exam twice, but we do not want really a toc
%------------------------------------- Custom Title Page ----------------

%------------------------------------- Custom Header ----------------
\pagestyle{headandfoot}     %% from exam.cls
\headrule
\ifprintanswers
\header{}{\textcolor{dkred}{\textbf{\Huge MUSTER}}}{}
\else
\header{Name:}{Matrikelnummer:}{}
\fi
\footrule
%% footer-definition for single-sided printing (empty backside)
%\footer{$LVKURZ$$if(SEMESTER)$, $SEMESTER$$endif$}{Klausur $if(PART)$$PART$ \xspace$endif$ $NR$}{Seite \thepage\ von \numpages}
%% footer-definition for double-sided printing
\lfoot{\oddeven{Seite \thepage\ von \numpages}{$LVKURZ$$if(SEMESTER)$, $SEMESTER$$endif$}}
\rfoot{\oddeven{$LVKURZ$$if(SEMESTER)$, $SEMESTER$$endif$}{Seite \thepage\ von \numpages}}
\cfoot{Klausur $if(PART)$$PART$ \xspace$endif$ $NR$}
%------------------------------------- Custom Header ---------------------

%------------------------------------- Questions --------------------------------
\providecommand{\theMyQuestionTitle}{}
\providecommand{\myQuestion}[2][0]{\vskip11pt\renewcommand{\theMyQuestionTitle}{#2}\question[#1]{\ }\vskip5pt}
\qformat{\textbf{\textcolor{headcolor}{\Large $if(PART)$$PART$.$endif$\thequestion: \theMyQuestionTitle\hfill \fbox{\thepoints}}}}  % add only question points
\boxedpoints    % no effect w/ qformat
\pointpoints{Punkt}{Punkte}
%------------------------------------- Questions --------------------------------



